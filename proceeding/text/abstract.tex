In gamma-ray astronomy, a variety of data formats and proprietary software have been traditionally used, often developed for one specific mission or experiment. Especially for ground-based imaging atmospheric Cherenkov telescopes (IACTs), data and software have been mostly private to the collaborations operating the telescopes. However, there is a general movement in science towards the use of open data and software. In addition, the next-generation IACT instrument, the Cherenkov Telescope Array (CTA), will be operated as an open observatory.

We have created a Github organisation at \ogragithub where we are developing high-level data format specifications. A public mailing list was set up at \ogralist and a first face-to-face meeting on the IACT high-level data model and formats took place in April 2016 in Meudon (France). This open multi-mission effort will help to accelerate the development of open data formats and open-source software for gamma-ray astronomy, leading to synergies in the development of analysis codes and eventually better scientific results (reproducible, multi-mission).

This writeup presents this effort for the first time, explaining the motivation and context, the available resources and process we use, as well as the status and planned next steps for the data format specifications. We hope that it will stimulate feedback and future contributions from the gamma-ray astronomy community.
